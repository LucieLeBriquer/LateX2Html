\newpage
\section{Shell}

\subsection{Quelques commandes}
\begin{itemize}
    \item \verb|man -f| : recherche dans tous les registres
    \item existence de \verb|rmdir| au lieu de \verb|rm -rf|
    \item \verb|wc| : wildcard, nombre de lignes, mots, bytes
    \item \verb|env| affiche l'ensemble des variables environnements, une variable créée n'est pas commune à plusieurs shell, il faut l'export si on veut pouvoir l'utiliser
    \item \verb|more| : pagine pour un long fichier, et on peut utiliser \verb|/| pour rechercher comme dans \verb|vim|
    \item \verb|head -n| : récupère les \verb|n| premières lignes 
    \item \verb|head -c| : récupère les \verb|n| premiers charactères
    \item \verb|tail| : fonctionne de la même façon
    \item \verb|grep| : recherche un motif, \verb|-v| pour afficher les lignes qui ne contiennent pas, \verb|-i| pour ignorer la casse, \verb|-n| pour afficher aussi les numéros lignes contenant le motif
    \item \verb|sed 's/motif/remplacement/g'| : remplace toutes les occurences de \verb|motif| par \verb|remplacement|
    \item \verb|chmod| pour changer les droits
    \item \verb|touch| créé un fichier, \verb|touch -t 07151997 fichier| pour modifier la date
    \item \verb|for i in `seq 0 9` ; do ... ; done|
    \item \verb|ln -h| pour créer un hardlink, \verb|ln -n| pour un lien symbolique
    \item \verb|ls -t| pour trier par ordre de dernière modification
    \item \verb|find . \( -type d -o -type f \)| trouver tous les dossiers et fichiers réguliers
\end{itemize}

\subsection{Sorties et écriture de fichiers}
\begin{itemize}
    \item \verb|stdin| est l'entrée standard de la commande lancée, une commande peut écrire sur deux types de sorties \verb|stdout| (sortie standard) et \verb|stderr| (sortie d'erreur)
    \item \verb||| branche la sortie standard sur l'entrée standard
    \item \verb|>| pour rediriger la sortie standard
    \item \verb|2>| pour rediriger la sortie d'erreur
    \item les deux peuvent s'enchaîner
    \item on peut aussi faire \verb|grep bl < fichier|
    \item \verb|>>| ajoute au fichier plutôt que de l'écraser
    \item \verb|<<| écrire jusqu'au mot choisi :
        \begin{verbatim}
        cat << FIN
        > ...
        > ...
        > FIN
        \end{verbatim}
    \item \verb|2<&1| pour rediriger la sortie d'erreur sur la sortie standard, pour des raisons techniques pour tout rediriger dans un fichier \verb|ls -l > fichier 2>&1|
\end{itemize}

\subsection{Remarques}
\begin{itemize}
    \item \verb|'  '| inhibe tout contrairement à \verb|"  "| pour lequel \verb|"$truc"| affiche \verb|machin|
    \item fichier magic pour écrire des conditions recherchées sur un fichier
        \begin{verbatim}
        offset  type    motif   message
        \end{verbatim}
        compiler ce fichier avec \verb|file -C -m fichier_magique| puis tester le fichier avec
        \\ \verb|file -m magic_file monfichier|
\end{itemize}
